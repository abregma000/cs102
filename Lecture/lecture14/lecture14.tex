
\documentclass[xcolor={dvipsnames}]{beamer}
\usepackage{amsmath,amsfonts,amssymb,pxfonts,eulervm,xspace}
\usepackage{graphicx}
 \usepackage{multimedia}
\usepackage{media9}
\usepackage{minted}
\usepackage{mathtools}

\usepackage{animate}

\usetheme{ccnycrest}


\begin{document}

\title{ CS102: Encapsulation and Overloading}
\author{Hannah Aizenman}
\date{haizenm00@ccny.cuny.edu}


\begin{frame}
	\titlepage
\end{frame}


\begin{frame}[fragile]{Account Object}
\begin{block}{Declaration}
\begin{minted}{c++}
class Account{
    public:
        std::string accountNumber;
        double balance;
};
\end{minted}
\end{block}
\pause
\begin{block}{Constructor}
\begin{minted}{c++}
Account::Account(std::string aN, double b){
    accountNumber = aN;
    balance = b;
}
\end{minted}
\end{block}
\end{frame}


\begin{frame}[fragile]{Using Account Object}
\begin{minted}{c++}
int main(){
    Account a1("1234", 100);
    a1.accountNumber = "4567";
    a1.balance -= 200;
    return 0;
}
\end{minted}
\pause
\begin{center}
\huge
No checks to changing information!
\end{center}
\end{frame}

\begin{frame}[fragile]{Encapsulated Account Object}
\begin{block}{Declaration}
\begin{minted}{c++}
class Account{
    private:
        std::string accountNumber;
        double balance;
    public: 
        ....
        double changeBalance(double value);
        double getBalance();
};
\end{minted}
\end{block}
\pause
\begin{block}{Constructor}
\begin{minted}{c++}
Account::Account(std::string aN, double b){
    accountNumber = aN;
    balance = b;
}
\end{minted}
\end{block}
\end{frame}

\begin{frame}[fragile]{Balance Method}
\begin{minted}{c++}
double Account::changeBalance(double value){
     if ((value<0) && (balance<std::abs(value))){
         throw value;
    }
    balance += value;
    return balance;
}     
\end{minted}
\end{frame}

\begin{frame}[fragile]{Use operators directly}
\begin{minted}{c++}
int main(){
    Account test2(name, account, 200);
    cout<<"Test overload"<<endl;
    cout<<"old balance: "<<test2.getBalance()<<endl;
    test2 -= 100;
    cout<<"new balance: "<<test2.getBalance()<<endl;
    return 0;
}
\end{minted}
\end{frame}

\begin{frame}[fragile]{Operator Overloading}
\begin{block}{Declaration}
\begin{minted}{c++}
class Account{
    private:
        std::string accountNumber;
        double balance;
    public:
        double changeBalance(double value);
        double operator-=(double value);
};
\end{minted}
\end{block}
\pause
\begin{block}{Overloaded Operator Method}
\begin{minted}{c++}
double Account::operator-=(double value){
     changeBalance(-1*value);
     return balance;
}
\end{minted}
\end{block}
\end{frame}

\begin{frame}[fragile]{Friend Functions}
\begin{minted}{c++}
int main(){
    string name1 = "1234";
    string name2 = "4212";
    Account test1(name1, 100);
    Account test2(name2, 200);
    cout<<test1==test2<<endl;
    return 0;
}
\end{minted}
\end{frame}

\begin{frame}[fragile]{Friend Function}
\begin{block}{Declaration}
\begin{minted}{c++}
class Account{
    private:
        std::string accountNumber;
        double balance;
    public:
        friend bool operator==(Account a1, Account a2);
};
\end{minted}
\end{block}
\pause
\begin{block}{Overloaded Operator Friend}
\begin{minted}{c++}
bool operator==(Account a1, Account a2){
	return ((a1.accountNumber==a2.accountNumber) && 
                       (a1.balance == a2.balance));
\end{minted}
\end{block}
\end{frame}

\begin{frame}{Method vs. Friend Function}
	\begin{description}
		\item[operands] object on which an operation is performed\\A and B in A+ B
		\item[relational operators] == , \textless=, \textless, \textgreater, \textgreater=, !=
		\item[mathematical operators] +, -, /, *
		\item[accumulators] +=, -=, *=. /=
		\item[friend] Operands \textbf{are not} modified
		\item[method] Operands \textbf{are} modified
			\begin{itemize}
				\item calling operand is implicit
				\item second operand is arg to function
				\item used for accumulators
			\end{itemize}
		
			
	\end{description}
\end{frame}

\begin{frame}[fragile]{Arrays of Objects}
\begin{minted}{c++}
int main(){
    string name1 = "1234";
    string name2 = "4212";
    Account test1(name1, 100);
    Account test2(name2, 200);
    Account accountList[2] = {test1, test2};
    return 0;
}
\end{minted}
\end{frame}
\end{document}
