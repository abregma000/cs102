
\documentclass[xcolor={dvipsnames}]{beamer}
\usepackage{amsmath,amsfonts,amssymb,pxfonts,eulervm,xspace}
\usepackage{graphicx}
 \usepackage{multimedia}
\usepackage{media9}
\usepackage{minted}
\usepackage{mathtools}

\usepackage{animate}

\graphicspath{{./figures/}}
\usetheme{ccnycrest}


\begin{document}

\title{ CS102: Arrays}
\author{Hannah Aizenman}
\date{haizenm00@ccny.cuny.edu}


\begin{frame}
	\titlepage
\end{frame}


\begin{frame}[fragile]{Array}
	\begin{block}{Array named "numbers"}
	\begin{table}
	\Huge
	\begin{tabular}{|c|c|c|c|c|}
	\hline
	2  & 4 & 6 & 8 & 10\\
	\hline
	\end{tabular}
	\end{table}
	\end{block}
	\pause
	\begin{block}{C++ implementation}
	\begin{minted}{c++}
	int numbers[5] = {2, 4, 6, 8, 10};
	\end{minted}
	\end{block}
\end{frame}

\begin{frame}[fragile]{Generic form of specifying arrays}
\begin{center}
	arrays can contain values of any type
\end{center}
\begin{block}{Declaration}
	\begin{minted}{c++}
	type name[length];
	\end{minted}
\end{block}
\begin{block}{Assignment}
	\begin{minted}{c++}
	type name[length] = {sequence of values};
	\end{minted}
	\end{block}
\end{frame}

\begin{frame}[fragile]{Sample Arrays}
	\begin{center}
		arrays can contain values of any type
	\end{center}
	\begin{minted}{c++}
		int numbers[5] = {2, 4, 6, 8, 10};
		double grades[3] = {3.5, 2.3, 1.0};
		char hello[6] = {'h', 'e', 'l', 'l', 'o', '\0'};
		bool mask[7] = {1, 1, 0, 0, 1, 1, 0};
	\end{minted}
\end{frame}



\begin{frame}{Array Indexing}
	\begin{block}{int numbers[5]}
	\begin{table}
	\Huge
	\begin{tabular}{|c|c|c|c|c|c|}
	\hline
	value & 2  & 4 & 6 & 8 & 10\\
	\hline
	{\color{red} index} &  {\color{red} 0}   &  {\color{red} 1}  &   {\color{red} 2}  &  {\color{red} 3}  &  {\color{red} 4} \\
	\hline
	\end{tabular}
	\end{table}
	\end{block}
	\pause
	\begin{block}{For element e in array numbers}
		\begin{description}
			\item[value] \textbf{what} the element is
			\item[index] \textbf{where} the element is 
		\end{description}
	\end{block}
\end{frame}

\begin{frame}[fragile]{Manipulating values using indexing}
	\begin{block}{General form}
	\begin{minted}{c++}
		name[index]
		name[index] = value;
	\end{minted}
	\end{block}
	\pause
	\begin{block}{Printing out the value at index 3}
	\begin{minted}{c++}
		cout<<numbers[3]<<endl;
	\end{minted}
	\end{block}
	\pause
	\begin{block}{Assign a value to index 4}
	\begin{minted}{c++}
		numbers[4] = 42;
	\end{minted}
	\end{block}
\pause
	\begin{block}{Input an element to index 2}
	\begin{minted}{c++}
		cin>>numbers[2];
	\end{minted}
	\end{block}
\end{frame}

\begin{frame}[fragile]{2D Array}
	\begin{block}{Array named "numbers"}
	\begin{table}
	\Huge
	\begin{tabular}{|c|c|c|c|c|}
	\hline
	2  & 4 & 6 & 8 & 10\\
	\hline
	3  & 5 & 7 & 9 & 11\\
	\hline
	\end{tabular}
	\end{table}
	\end{block}
	\pause
	\begin{block}{C++ implementation}
	\begin{minted}{c++}
	int numbers[2][5] = {{2, 4, 6, 8, 10},
	                     {3, 5, 7, 9, 11}};
	\end{minted}
	\end{block}
\end{frame}

\begin{frame}{2D Array Indexing}
	\begin{block}{int numbers[5]}
	\begin{table}
	\Huge
	\begin{tabular}{|c|c|c|c|c|c|}
	\hline
	 {\color{red} 0} & 2  & 4 & 6 & 8 & 10\\
	\hline
	 {\color{red} 1}  & 3  & 5 & 7 & 9 & 11\\
	\hline
	{\color{red}index} &  {\color{red} 0}   &  {\color{red} 1}  &   {\color{red} 2}  &  {\color{red} 3}  &  {\color{red} 4} \\
	\hline
	\end{tabular}
	\end{table}
	\end{block}
	\pause
	\begin{block}{For element e in array numbers}
		\begin{center}
		row index then column index
		\end{center}
	\end{block}
\end{frame}

\begin{frame}[fragile]{Manipulating values using indexing}
	\begin{block}{General form}
	\begin{minted}{c++}
		name[row][column]
		name[row][column] = value;
	\end{minted}
	\end{block}
	\pause
	\begin{block}{Printing out the value at row index 0 and column index 3}
	\begin{minted}{c++}
		cout<<numbers[0][3]<<endl;
	\end{minted}
	\end{block}
	\pause
	\begin{block}{Assign a value to row index 1 and column index 4}
	\begin{minted}{c++}
		numbers[1][4] = 42;
	\end{minted}
	\end{block}
	\pause
	\begin{block}{Input an element to row index 0 and column index 2}
	\begin{minted}{c++}
		cin>>numbers[0][2];
	\end{minted}
	\end{block}
\end{frame}

\begin{frame}[fragile]{Generic: Arrays as a function argument}
	\begin{block}{prototype}
	\begin{minted}{c++}
		function_type function_name(array_type array_name[]);
	\end{minted}
	\end{block}
	\pause
	\begin{block}{implementation}
	\begin{minted}{c++}
		function_type function_name(array_type array_name[]){
			//function body
		}
	\end{minted}
	\end{block}
\end{frame}

\begin{frame}[fragile]{Arrays as a function argument}
	\begin{block}{prototype}
	\begin{minted}{c++}
		int max(int arr[], const int LENGTH);
	\end{minted}
	\end{block}
	\pause
	\begin{block}{implementaton}
	\begin{minted}{c++}
int max(int arr[], const int LENGTH){
    int maxElement = arr[0];
    for (int i=0; i<LENGTH; i++){
        if (arr[i]>maxElement){
            maxElement=arr[i];
        }
    }
    return maxElement; 
}
	\end{minted}
	\end{block}
\end{frame}

\begin{frame}[fragile]{Generic: 2D Arrays as a function argument}
	\begin{block}{prototype}
	\begin{minted}{c++}
		function_type function_name(array_type array_name[][]);
	\end{minted}
	\end{block}
	\pause
	\begin{block}{implementation}
	\begin{minted}{c++}
		function_type function_name(array_type array_name[][]){
			//function body
		}
	\end{minted}
	\end{block}
\end{frame}

\begin{frame}[fragile]{Arrays as a function argument}
	\begin{block}{prototype}
	\begin{minted}{c++}
		int max(int arr[][], const int ROWS, const int COLUMNS);
	\end{minted}
	\end{block}
	\pause
	\begin{block}{implementaton}
	\begin{minted}{c++}
int max(int arr[][], const int ROWS, const int COLUMNS){
    int maxElement = arr[0][0];
    for(int i=0; i<ROWS; i++){
        for(int j=0; j<COLUMNS; j++){
            if (arr[i][j]>maxElement){
                maxElement=arr[i][j];
            }
        }
    }
    return maxElement;
}
	\end{minted}
	\end{block}
\end{frame}

\end{document}

