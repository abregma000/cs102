\documentclass{article}
\usepackage{graphicx}
\usepackage[margin=1in]{geometry}
\usepackage{listings}
\usepackage{amsmath}

\begin{document}

\title{CS102: Week 6}

\maketitle

\section*{Array Manipulation}
Given the following code:
\begin{lstlisting}{c++}
	const int LEN = 8;
	int arr[LEN] = {0,1,1,2,3,5,8,13};
\end{lstlisting}
Write a program to:
\begin{enumerate}
	\item print out all the elements in arr
	\item add 2 to every element e in arr
	\item replace every element e in arr with a+2
	\item print out all the elements in arr to verify replacement
\end{enumerate}

\section*{User Input}
Write a program that:
\begin{enumerate}
	\item asks users how many elements N they want in an array
	\item asks the user to enter N values and places those values into the array
	\item prints out the elements of the array
\end{enumerate}

\section*{Indexing}
Write a program that
\begin{enumerate}
	\item prints out the elements of an array backwards
	\item prints out every other element of the array
\end{enumerate}
\pagebreak
\section*{Function: Statistics}
Write the following functions and a main program to test them:
\begin{enumerate}
	\item write a function that computes and returns the sum of the elements in the array. Use the following prototype:
	\begin{verbatim}
	double sum( double arr[], const int LENGTH)l
	}
	\end{verbatim}
	\item write a function that computes and returns the mean of the elements in the array. Use the following prototype:
	\begin{verbatim}
	double mean( double arr[], const int LENGTH)l
	}
	\end{verbatim}
\end{enumerate}

\section*{Function: Counting}
Write the following functions and a main program to test them:
\begin{enumerate}
	\item write a function that returns the number of elements in the array divisible by 2
	\item write a function that returns the number of elements in the array divisible by n
\end{enumerate}

\section*{Function: Sorting}
Write a function to sort the elements of the array and a main program to test your implementation. Use the following prototype:
\begin{verbatim}
	int sort(int arr[]; const int LENGTH);
\end{verbatim}

\end{document}