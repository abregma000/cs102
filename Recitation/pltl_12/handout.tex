\documentclass{article}
\usepackage{graphicx}
\usepackage[margin=1in]{geometry}
\usepackage{listings}
\usepackage{amsmath}
\usepackage{mathtools}

\begin{document}

\title{CS102: Week 11}

\maketitle
\section*{Account Class: Public}
Using the account class with this declaration: 
\begin{lstlisting}
class Account{
    public:
        std::string accountNum;
        double balance;
}
\end{lstlisting}
\begin{description}
    \item[implement the constructor:]Account(std::string name, double b) 
    \item[print out the following inside main:] ''Account number: \textit{accountNum} Balance: \textit{balance}''
\end{description}

\section*{Account Class: Private}
Using the account class with this declaration:
\begin{lstlisting}
class Account{
    public:
        Account(std::string name, double b);
    private:
        std::string accountNum;
        double balance;
};
\end{lstlisting}
implement the following methods:
\begin{description}
    \item [Account(std::string name, double b)]
        constructor
    \item [void display()]
     prints out ''Account number: \textit{accountNum}
     Balance: \textit{balance}''
    \item[changeBalance(double amount)]
    change the balance by the amount and throw an error if the amount
    is greater than the balance
\end{description}

\section*{Operator Overloading}
Overload the following operators in the account class described above (private):
\begin{description}
    \item[+] returns the balance + amount, but don't change the balance
    \item[+=] permanently adds the amount to the balance
\end{description}

\section*{History}
Add the following to the account class described above(private):
\begin{description}
    \item[double history[10];] attribute to record the last 10 entries
    \item[displayHistory()] print out the elements in history
    \item[updateHistory(double transaction)] method to add new
    entries. History[0] is the newest transaction.
\end{description}
 
\section*{Merge}
Add the following method to the account class described above(private):
\begin{description}
    \item[merge(Account B)] merges the balance from B into A
\end{description}

\section*{Friend Function}
Implement a friend function to merge two accounts into a new account.
Use the following declaration and constructors:
\begin{lstlisting}
class Account{
    public:
        Account();
        Account(std::string name, double b);
        double getBalance();
        friend Account merge(std::string name, Account a1, Account a2);
    private:
        std::string accountNum;
        double balance;
};
Account::Account(){
}
Account::Account(std::string name, double b){
    accountNum = name;
    balance = b;
}
\end{lstlisting}
Test your function using the following statements in main:
\begin{lstlisting}
string n1 = ''1234'';
string n2 = ''3245'';
Acount test1(n1, 200);
Account test2(n2, 300);
Account testMerge = merge(''2334'', test1, test2);
cout<<testMerge.accountNumber<<'' ''<<testMerge.getBalance()<<endl;
\end{lstlisting}
\end{document}