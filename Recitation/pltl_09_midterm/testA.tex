\documentclass[addpoints,12pt]{exam}
\usepackage[margin=1in]{geometry}
\usepackage{listings}
\usepackage{multicol}
\usepackage{tabularx}
\newcounter{matchleft}
\newcounter{matchright}

\newenvironment{matchtabular}{%
  \setcounter{matchleft}{0}%
  \setcounter{matchright}{0}%
  \tabularx{\textwidth}{%
    >{\leavevmode\hbox to 1.5em{\stepcounter{matchleft}\arabic{matchleft}.}}X%
    >{\leavevmode\hbox to 1.5em{\stepcounter{matchright}\alph{matchright})}}X%
    }%
}{\endtabularx}

\begin{document}
\header{CS102}{Midterm Version A}{}

\begin{center}
\fbox{\fbox{\parbox{5.5in}{\centering
Save each program lastname\underline{\hspace{.25cm}}firstname\underline{\hspace{.25cm}}q\#.cpp and save all files to a folder named lastname\underline{\hspace{.25cm}}firstname. You will be submitting this folder as your exam answer. 
\\
\textbf{Show all work}. \\

Credit \textbf{will not} be given if work is not shown.}}}
\end{center}
\vspace{0.1in}
\makebox[\textwidth]{Name and section:\enspace\hrulefill}
\begin{center}
\gradetable[h][questions]
\end{center}

\begin{questions}

\question [15]
Using the provided template (question1.cpp), write, run, and test a program that for all numbers between 100 and 200 (inclusive) prints out the number and
\begin{itemize}
	\item print \textbf{Crackle} for multiples of 2
	\item print \textbf{Pop} for multiples of 7
	\item print \textbf{CracklePop} for multiples of 2 and 7
\end{itemize}

\question[25]
The following is an approximate conversion formula for converting Fahrenheit temperature to Celsius temperatures:
\begin{equation}
Celsius = (Fahrenheit - 30)/20
\end{equation}
The following is the real conversion formula for converting Fahrenheit to Celsius:
\begin{equation}
Celsius =  \frac{5}{9}(Fahrenheit - 32)
\end{equation}
The approximation diverges from the real  number as temperature increases. Using these formulas, and starting with a Fahrenheit temperature of 0 degrees, write a C++ program that finds the temperature at which the approximate equivalent Celsius temperature differs from the exact equivalent value by more than four degrees.


\question[25]
Implement a function to solve for the roots of a quadratic equation. Your function should print an error message when the roots are imaginary and return the \textbf{positive root}.  Use the provided prototype:
\begin{lstlisting}
	double quadratic(int a, int b, int c);
\end{lstlisting}
Use the provided template(question3.cpp) to write, run, and test the function.

\pagebreak
\question[35]
A mathematical set is a collection of non repeating elements. Write a function to test whether an array of arbitrary size containing only digits is a set. The function should return true when the array \textbf{is a set}. You may not use a sort function. Use the following prototype:
\begin{lstlisting}
	bool isSet(int array[], const int LENGTH);
\end{lstlisting}

For example:\\
\{1, 2, 3\} returns \textbf{true}\\
\{1, 2, 2\} returns \textbf{false}\\

Write, run, and test a program to test the the function. The program must also test that the array only contains 0-9.


\end{questions}
\end{document}