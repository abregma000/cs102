\documentclass{article}
\usepackage{graphicx}
\usepackage[margin=1in]{geometry}
\usepackage{listings}
\usepackage{amsmath}
\usepackage{mathtools}

\begin{document}

\title{CS102: Week 13}

\maketitle
\section*{Memory Allocation}
Using the sizeof() operator determine the number of bytes your computer uses to store:
\begin{enumerate}
	\item an integer, a character, and a double-precision number
	\item the \textbf{address} of an integer, a character, and a double-precision number
\end{enumerate}
Hint: sizeof(*int) can be used to determine the number of memory bytes used for a pointer to an integer.) Would you expect the size of each address to be the same? Why or why not?
\section*{Pointers and References}
Write a C++ program that includes the following code:
\begin{lstlisting}
int b;
int& a = b;
a = 10;
cout<<"a="<<a<<" b="<<b<<endl;

int d;
int *c = &d;
*c = 10;
cout<<"c="<<c<<" d="<<d<<endl;
\end{lstlisting}
Explain what each line does. Then modify the program so that a and c are set based on the 2nd and 3rd commandline arguments (hint: substitute a variable for 10); 

\section*{Pointer Operations}
Write a program that declares three one-dimensional arrays named miles, gallons, and mpg. Each array should be capable of holding 10 elements.
\begin{description}
\item[miles] 240.5, 300.0, 189.6, 310.6, 280.7, 216.9, 199.4, 160.3, 177.4, 192.3. 
\item[gallons] 10.3, 15.6, 8.7, 14, 16.3, 15.7, 14.9, 10.7, 8.3,  8.4. 
\item[mpg] mpg[i] = miles[i] / gallons[i]
\end{description}
Use pointers when calculating and displaying the elements of the mpg array.

\section*{Structs}
Write a C++ program that accepts a user-entered date. Have the program calculate and display the date of the next day. For the purposes of this exercise, assume all months consist of 30 days. Then modify the program written in Exercise 6a to account for the actual number of days in each month.



\end{document}